\documentclass{beamer}

% package hyperref needs these to be set here

\author{Pritvi Jheengut @zcoldplayer}
\subject{DevConMU - Open Data}

\usepackage[size=a4, orientation=landscape, scale=3.5]{beamerposter}

\usepackage{lmodern}
\usepackage{fontenc}[T1]
\usepackage{inputenc}[utf8]
\usepackage{hyperref}[pdfpagelabels=true,bookmarks=true,
  unicode=true]
\usepackage{pgf}
%\usepackage{verbatim}

\usepackage{listings}
\usepackage{xy}[all]%% License to be introduced


%  Eliminate errors such as
%  LaTeX Font Warning: Font shape `T1/cmss/m/n' in size <4> not available
%  LaTeX Font Warning: Size substitutions with differences up to 1.0pt 

\mode<presentation>{
  
  \useoutertheme{infolines}
  \usecolortheme{whale}
  
  \definecolor{ncurses}{HTML}{B7DFFF}
  \definecolor{darkblue}{HTML}{253789}
  \definecolor{light}{HTML}{A2CDEE}
  \definecolor{confgreen}{HTML}{3CA387}
  
  \setbeamercolor{Title bar}{fg=black!50}
  \setbeamercolor{frametitle}{parent=Title bar}
  \setbeamercolor{titlelike}{parent=light,bg=confgreen,fg=white}
  \setbeamercolor{block title}{bg=darkblue,fg=light}
  
  \setbeamercovered{transparent,dynamic}
  
  \setbeamertemplate{blocks}[rounded][shadow=true]
  \setbeamertemplate{background canvas}[vertical shading]
                    [top=confgreen!35,middle=light!5,bottom=red!50]

  \mode<handout>{\beamertemplatesolidbackgroundcolor{black!50}}
                    
}

\begin{document}

\section{Definition and Introduction to Open Data}

\title[DevConMU - Open Data]{Open Data in the Government of 
  Mauritius. Finally!!!}
\subtitle{Three years ago, the World Bank introduced and
  leveraged the  use of \\
  Open Data onto the Government of Mauritius. \\
  A great era of Open Data is looming forward, \\
  maybe the next big pillar of our economy. \\
  This session will cover a broad definition of Open Data,\\
  current platforms delivering Open Data by other governments in 
  the World. \\
  An overview of how to use Open Data and exploiting it.\\
  A wide coverage and importance of licensing and intellectual
  property.  
}

\date[Developers Conference - Educator 2]{Developers Conference
  -  19 May 2018}

\maketitle

\subsection{CopyLeft License}

\frame{
  \frametitle{Copyleft License Attribution}

  Made with love using beamer, LaTeX and git.
  you can view clone at \url{https://github.com/Pritvi-DevConMRU-MSCC/OpenDataInitiativeMru/tree/OpenData-GovMRU}

  
  \begin{alertblock}{This work is licensed under the LaTeX Project
      Public License.}
    To view a copy of this license, visit \\
    \url{https://www.latex-project.org/lppl.txt}    
  \end{alertblock}
  
  \begin{alertblock}{This work is licensed under the Creative
      Commons Attribution 4.0 International License.}
    To view a copy of this license, visit \\
    \url{http://creativecommons.org/licenses/by/4.0/} or \\
    
    send a letter to \\
    Creative Commons, \\
    PO Box 1866, \\
    Mountain View, \\
    CA 94042, \\
    USA. \\
  \end{alertblock}
}

\subsection{Free and Open Source Software}

\frame{
  \frametitle{FOSS - Free Source Software}
  
  The origins of Open Data dates back since the Free Software
  Movement.
  
  \pause
  
  \begin{block}{Free Software}
    \begin{itemize}
    \item The Free Software Movement began in the 1980's
    \item Free Software cannot exist without a License
    \item The first FREE SOFTWARE LICENSE :: The GPL,GNU General
      Public License
    \end{itemize}
  \end{block}
  
  \pause
  
  \begin{block}{What do you understand by FREE Software}
    we mean that it respects the users' and the developers essential
    freedoms:
    \begin{itemize}
    \item the freedom to run it,   
    \item to study and
    \item change it, and
    \item to redistribute copies with or without  changes. 
    \end{itemize}
    
  \end{block}

  \pause
  
  \begin{fact}
    This is a matter of freedom, not price, so think of free speech,
    not free beer.
  \end{fact}
  
}

\frame{
  \frametitle{FOSS - Open Source Software}

  \begin{block}{\url{https://opensource.org/}}
\begin{quote}{}
    The Open Source Initiative is celebrating its 20th Anniversary
    in 2018.\\
    OSI was formed in 1998 as an educational, advocacy, and
    stewardship organization at this important moment in the
    history of collaborative development.
    \end{quote}
  \end{block}
  
  \begin{block}{The Open Source Movement Initiative started in the
      late 90's with the most famous example::}
    The death of Netscape and Mozilla  as a phoenix.
  \end{block}

}

\frame{
  \frametitle{Open VS Free Software License! Who wins?}
  
  \begin{block}{Confused}
    There exist not much difference between an Open and a Free
    Software License.
  \end{block}
  
  \pause
  
  \begin{block}{Famously pointed by Richard Stallman }
    In the blog post of Why Open Source misses the point of Free
    Software
    \url{https://www.gnu.org/philosophy/open-source-misses-the-point.html.en}
    
  \end{block}

  \pause
  
  \vspace{2cm}
  \begin{quote}
    \huge{
      Open source is a development \\
      methodology;\\
      
      \pause
      
      Free software is a \\
      social movement.
    }
  \end{quote}
  
}

\section{Open Data}
\subsection{Definition of Open Data}

\frame{
  \frametitle{What is Open Data after all?}
  
  \begin{block}{Open Data is characterised by the fact that }
    data should be available in a suitable format to everyone to
    \pause with an unlimited right for use, reuse  and
    \pause republish, redistribute as they wish without
    restrictions from copyright patents or other mechanisms of
    control. 
  \end{block}

  \pause
  
  \begin{Definition}
    Open Data is also usually defined as DATA that can be easily
    retrieved, read and processed by both \\
    \huge { \pause HUMANS and MACHINES.}
  \end{Definition}

  \pause
  
  \begin{quote}
    \huge{A completely Open Data should be \\
      characterised as free of any cost.}
  \end{quote}

  \begin{alertblock}{source:}
    McKinsey Global Institute Analysis
  \end{alertblock}

}

\subsection{Why use Open Data?}

\frame{
  \frametitle{Advantages of Open Data}

  \begin{block}{Advantages and Benefits of Open Data for the Republic
      of Mauritius - Part 1}
    
    \begin{itemize}
    \item Commissions Open Standards and Collaboration when royalty
      free Open Data is made available through an Open Data Portal.
    \item Innovative Entrepreneurial, Business and Economic valuable
      resources help companies to define new products and services
      thus increases and encourages competitivity, collaboration and
      improvements.
    \item Training Environment promotes creativity and lowers
      barriers to transfer of knowledge enabling to build advanced
      technical skills.
    \item Enables the creation and upgrade of local ICT skills as
      well as startups via open collaboration thus lowering
      cost for local ICT development with the  adoption of FLOSS.
      
    \end{itemize}
    
  \end{block}
  
}

\frame{
  \frametitle{Advantages of Open Data}
  
  \begin{block}{Advantages and Benefits of Open Data for the Republic
      of Mauritius - Part 2}
    
    \begin{itemize}
    \item Reduce search and transaction costs via a common Open API
      while also emphasising the promotion of social inclusion,
      participation and transparency
    \item Analysis and accountability publishing by NGO's, Civil
      Society Organisations and media outlets of Open Data will
      permit more Data exchange and transparency on top of feeding
      the results into better governmental policy.
    \item Increases Governmental agencies efficiency by empowering
      the acquisition, exchange, reuse and sharing of Datasets within
      governmental bodies.
    \end{itemize}  
    
    \huge Sharing is caring.
    
  \end{block}
  
}

\subsection{Open Datasets Portals}

\frame{
  \frametitle{Open Data Portal}
  
  \begin{block}{Dataset or Data Set}
    There exist no formal definition of a Dataset in Open Data
    parlance. A Dataset can be understood as matrix of data
    whereby the column represents a particular variable and the
    row a given datum.\\
    All Open data formats should be defined with an Open Standard
    metadata on top of data or else it is not machine readable nor
    might it be human readable.    
  \end{block}

  \begin{block}{The Open Data Portal}
    An Open Data portal is a repository where several Datasets are
    stored using an Open Standard framework such that a user has
    been attributed without any discrimination the capacity to
    acquire and retrieve data such that it is relevant, exact,
    correct, reusable, royalty free and unrestricted.
  \end{block}
  
}
    
\frame{
  \frametitle{Governmental Open Data Providers - Part 1}
  
  \begin{block}{\url{https://data.europa.eu/euodp/en/home}}
    The European Union Open Data Portal provides access to open data
    published by EU institutions and bodies.
  \end{block}

  \begin{block}{\url{https://www.europeandataportal.eu/}}
    The European Data Portal harvests the metadata of Public
    Sector Information available on public data portals across
    European countries.
  \end{block}
  
  \begin{block}{\url{https://data.gov.au/}}
    Data.gov.au provides an easy way to find, access and reuse
    public data. Improve functionality based on user feedback.
    Government Data is encouraged to analyse, mashup and develop
    tools and applications to benefit all Australians. 
  \end{block}

}

\frame{
  \frametitle{Governmental Open Data Providers - Part 2}

  \begin{block}{\url{https://data.gov.in/}}
    This portal is intended to be used by Government of India to
    publish Datasets, documents, services, tools and applications
    collected by them for public use. It intends to increase
    transparency in the functioning of Government.
  \end{block}
    
  \begin{block}{\url{https://www.data.gov/}}
    The home of the U.S. Government's open data provides data,
    tools, and resources to conduct research, develop web and
    mobile applications, design data visualizations, and more.
  \end{block}
  
  \begin{block}{\url{https://open.canada.ca/en/open-data}}
    Portal of Canadian Open Data where anyone can learn how to work
    with Datasets, and see what people have done with open data
    across the country.
  \end{block}
  
}

\frame{
  \frametitle{Governmental Open Data Providers - Part 3}

  \begin{block}{\url{https://data.gov.uk/}}
    Find data published by central government, local authorities
    and public bodies to help you build products and services.
  \end{block}
  
  \begin{block}{\url{https://www.data.gouv.fr/fr/}}
    Partagez, am\'eliorez et r\'eutilisez les donn\'ees publiques
    avec la mission d'accompagner l'ouverture des donn\'ees
    publiques de l'Etat et des administrations.
  \end{block}

  \begin{block}{\url{https://data.govmu.org/}}
    OpenData Mauritius in line with the e-Government Strategy and
    the Open Data Policy, has implemented the National Open Data
    portal which houses and provides links to the Datasets of
    Government Agencies in an open format. This initiative empowers
    citizens and businesses for carrying out data-driven
    initiatives such as development of mobile apps, data analysis,
    creation of innovative products and research among others.
  \end{block}
    
}

\frame{
  \frametitle{Non Governmental Open Data Providers}

  \begin{block}{\url{http://opendatatoolkit.worldbank.org/en/}}
    The Open Government Data Toolkit help governments understand
    the precepts of Open Data to plan and implement an Open
    Government Data program.
  \end{block}
    
  \begin{block}{\url{http://opendata.cern.ch/}}
    Explore more than 1 petabyte of open data from particle physics!
  \end{block}
  
  \begin{block}{\url{https://data.world/}}
    The most meaningful, collaborative, and abundant data resource
    in the world.
  \end{block}
  
  \begin{block}{\url{https://data.cdp.net/}}
    The Carbon Disclosure Project Open Data Portal.
  \end{block}
  
  \begin{block}{\url{http://www.odatastore.com/}}
    Open Data Store for open, transparent, collaborative City
    platform. 
  \end{block}
  
  Further information available via
  \url{http://docs.getdkan.com/en/latest/introduction/dkan-sites.html}
}

\subsection{Overview of using Open Data}
    
\frame{
  \frametitle{Datasets}
  
  \begin{block}{Types of Available Datasets}
    There are several types of Datasets such as
    \begin{itemize}
    \item List of Emergency Numbers, Police Stations \& Hospital
    \item WIFI Hotspots, Internet Services \& ICT usage in business
    \item Total expenditure of the Judiciary
    \item Workers injured in work accidents
    \item Rates of allowances paid under social aid
    \item Average expenditure per tourist per night by country of residence
    \item Vital statistics rates \& Population growth in intercensal periods 
    \item Postcodes and Localities 
    \item Transport statistics; Rates of accident, fatality \& casualty
    \item Productivity and Competitiveness Indicators
    \item Wage Rate Indices
    \item Electors, voters and valid votes \& Candidates, and seats
      occupied in the national parliament
    \item Budgetary Central Government
    \item Forest plantations by type of plants
    \item Offence rate per 1000 population by type
    \end{itemize}
  \end{block}

}
    
\frame{
  \frametitle{Accessing Open Data and Cataloguing Open Data}
  
  \begin{block}{Getting Access to Open Data}
    Once the Datasets have been identified, the Open Data Portal can either
    permit anonymous access or a permitted user to gain access to a
    particular type of Open Data.
  \end{block}
  
  \begin{block}{Description, Identification and Open Data Documentation}
    
    \begin{itemize}
    \item Metadata is imperative using Open Standards to describe the Open
      Data.
    \item Taxonomy of Open Data allows it to become searchable and
      identifiable for further use. 
    \item A good Documentation of the Dataset is crucial so as to avoid
      public complaints becomes necessary while either the metadata or
      the portal of the Dataset should furnish such information.
    \end{itemize}
  \end{block}
    
}
    
\frame{
  \frametitle{Publishing Open Data}
  
  \begin{block}{Publishing of Open Data}
    Open Datasets are published using Open Data formats that are human
    and machine readable and for which the documentation is available
    for everyone to consult.
  \end{block}


  \begin{block}{Some common formats of Open Data are:}
    \begin{itemize}
    \item ASCII
    \item CSV
    \item JSON, GeoJSON
    \item XML
    \item RDF
    \item Tagged HTML / API
    \end{itemize}
  \end{block}

  \begin{block}{Found out which Open Data satisfying your needs}
    Once Open Data has been scouted out there are several steps into
    discovering the magic of Governmental Open Data.
  \end{block}
    
}
    
\frame{
  \frametitle{Using Open Data}
  
  \begin{block}{Using Open Data}
    \begin{itemize}
    \item Plotting Graphs, bar charts,... etc are the most convenient
      way of visualising Open Data and is one the easiest to understand.
    \item Make a table, pull the data into a spreadsheet or database
    \item Add the data on top of another layer of any GIS enabled program
      such as QGIS, OpenStreetMap
    \item Engage your processed Open Data by using social sharing
    \item Make blog post of the Open Data
    \item Enhance your CMS with Open Data
    \item Become a Data Journalist
    \item Organise Hackathon, BootCamps, talk about it.
    \end{itemize}

    Most Importantly, link your Open Data to the Governmental Open Data,
    this will surely aggrandise your own Datasets. This is win-win
    situation for all concerned parties.
    \url{https://eaves.ca/2010/04/14/case-study-open-data-and-the-public-purse/}

  \end{block}
  
}

\section{The Importance of Licensing and Intellectual Property}

\frame{
  \frametitle{Who Owns the Data?}
  
  \begin{block}{Licensing Data}
    Open Data becomes usable when published in a Machine and Human
    readable Data Format but this criteria does not justify its status
    as Open Data. \\
    Data is classified as Open Data only when it is licensed. \\
    The Terms and Conditions of the use of Open Data is governed by an
    Open License such as the Creative Commons Attribution 4.0
    International License. \\
    Once licensed, people and machines must be able to use the Open Data
    in any manner, including transforming, combining and sharing
    nonethelessly commercially too.
  \end{block}

  \begin{block}{Copyright of Open Data}
    Copyright is an area of Intellectual Property law which covering
    original creative works as such Data. In the case of the Government
    Open Data, copyright belong to the Government depending on the
    license it is using for its Data. There is also Laws which This is
    clearly very important.
  \end{block}
  
  Licensing allows copyright owners to permit approved use and reuse
  of their work, without relinquishing copyright fully. Licensing
  can permit both commercial and non-commercial reuse of a work,
  depending on the terms of the licence, and licences may last in
  perpetuity or for a specified period. The application of a licence
  does not mean that a copyright statement should not be applied
  to a work, and many licences such as Creative Commons suggest that
  the copyright holder is credited. Open Data is usually associated with
  an Open Licence such as CC-BY (Creative Commons Attribution Only)
  or a Publication Domain Dedication such as CC0.
  
}

\frame{
  \frametitle{Open Source Initiative vs Freedom of Information Act}
  
  \begin{block}{Open Source Initiative}

  \end{block}

  
  \begin{block}{Freedom Of Information Act}
    Freedom of Information will be a powerful tool at the hands of the
    citizens and promote transparency and accountability.

    The  main  difference  between  the  Open  Data  Initiative  and
    the  Freedom  of Information   Act intended   to   be   introduced
    by Government   according   the Government  Programme  2015 – 2019
    ,  is  that  Open  Data  proactively  looks  at datasets  that
    Government  can  make  publicly  available  whilst  the  Freedom
    of Information Act reacts to requests for data made by citizens.
    Hence, the Open by Default concept governs the Open Data Initiative.
    \end{block}

}
\end{document}



